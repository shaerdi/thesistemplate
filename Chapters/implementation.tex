\section{Simulationsframework}
\label{Kapitel:Implementierung}
Die Implementierung der zugrunde liegenden Gleichungen und ihre Diskretisierung wurde in \openfoam{} \cite{openfoam} realisiert.
\openfoam{} ist eine freie CFD Bibliothek für finite Volumen, in der zahlreiche Standardlöser schon implementiert sind und die es erlaubt auf eine einfache Art und Weise eigenen Code zu implementieren.\\
Für die Simulation der Viskoelastischen Fluide wurde ein Code von J.~Favero \cite{faveroOF} verwendet, der an die in dieser Arbeit verwendeten Modelle und Gleichungen angepasst wurde.

Die simulierten Mörtel weisen eine im Verhältnis zu den Strömungsgeschwindigkeiten hohe Viskosität auf. 
Die maximale Reynolds-Zahl die in in dieser Arbeit gemachten Simulationen auftritt ist im einstelligen Bereich, weshalb auf eine Turbulenz-Modellierung verzichtet und eine laminare Strömung vorausgesetzt wurde.
%
\subsection{Löser}
\subsubsection{simpleFoam}
Für die scherratenabhängigen Viskositätsmodelle wurde als Basislöser die schon in \openfoam{} eingebaute Version \codeemph{simpleFoam} des SIMPLE Algorithmus verwendet.
Diese erlaubt es unter verschiedenen schon implementierten Viskositätsmodellen zu wählen, oder diese zu modifizieren um ein eigenes Modell zu implementieren.\\
Dabei wird in der Datei \codeemph{transportProperties} als \codeemph{transportModel} das gewünschte Modell mit seinen Parametern hinterlegt:
%
\begin{lstlisting}
transportModel modifiedHerschelBulkley;

modifiedHerschelBulkleyCoeffs
{
    tau0        tau0  [ 0 2 -2 0 0 0 0 ] 343.0;
    k           k     [ 0 2 -1 0 0 0 0 ] 3.75;
    m           m     [ 0 0  1 0 0 0 0 ] 1000;
    n           n     [ 0 0  0 0 0 0 0 ] 0.848;
    nu0         nu0   [ 0 2 -1 0 0 0 0 ] 800;
}
\end{lstlisting}
%
\subsubsection{timeAdjustNonNewtIcoFoam}
Die Löser der in \openfoam{} implementierten Version des PISO Algorithmus heisst \codeemph{icoFoam}.
Diese Version unterstützt aber keine nicht Newtonsche Modelle und ist nicht in der Lage, den Zeitschritt an die maximal auftretende Geschwindigkeit anzupassen.\\
Um trotzdem in der Lage zu sein nicht Newtonsche, transiente Probleme zu lösen, wurde im Rahmen dieser Arbeit der Löser \codeemph{icoFoam} erweitert.

Analog zum Löser \codeemph{simpleFoam} wird im Set-up zusätzlich eine Instanz der Klasse \codeemph{transportModel} initialisiert, die dann in jedem Zeitschritt die kinematische Viskosität aktualisiert und zur Verfügung stellt:
%
\begin{lstlisting}
...
singlePhaseTransportModel fluid(U, phi);
...
while (runTime.run()) {
fluid.correct();
nu_ = fluid.nu();
}
...
\end{lstlisting}
%
Zusätzlich wurde in der Runtime Schleife die von \openfoam{} zur Verfügung gestellte Routine \codeemph{setDeltaT.H} eingefügt, um den $\Delta t$ in jedem Zeitschritt zu überprüfen und gegebenenfalls anzupassen.\\
Mit dieser Änderung können im \fileemph{controlDict} die Zeilen
%
\begin{lstlisting}
adjustTimeStep yes;
maxCo 0.5;
maxDeltaT 1;
\end{lstlisting}
%
eingefügt werden, um den Zeitschritt variabel zu machen.

Der resultierende Code \codeemph{timeAdjustNonNewtIcoFoam} ist in der Lage, sowohl mit nicht konstanten Viskositäten zu rechnen, als auch den Zeitschritt so zu wählen dass eine vom Nutzer bestimmte Courant Zahl \todo{Definition Courant Zahl} nicht überschritten wird.
%
\subsubsection{viscoelasticFluidFoam}
Der Löser für die viskoelastischen Simulationen wurde von J.~Favero implementiert. Der Code ist eine Implementation des DEVSS Algorithmus, der aber für die Berechnung der Schubspannungsterme auf eine separat implementierte Klasse zurückgreift.\\
Dieser Code wurde unverändert übernommen und ist hier nur der Komplettheit halber aufgeführt.
%
\subsection{Rheologische Modelle}
\subsubsection{Scherratenabhängige Modelle}
Die Implementation eigener Modelle wurde realisiert indem eine neue \cpp{} Klasse implementiert wurde, die von der in \openfoam{} vorhandenen Klasse \codeemph{viscosityModel} abgeleitet ist.
Diese neue Klasse muss die Funktionen \codeemph{nu()} und \codeemph{correct()} zur Verfügung stellen, die die kinematische Viskosität zurückgeben beziehungsweise neu berechnen.
Dabei kann auf die Scherrate $\gammap$ zurückgegriffen werden durch die von der Basisklasse zur verfügung gestellte Funktion \codeemph{strainRate()}. Das Einlesen von zusätzlichen Parametern geschieht dabei in der Funktion \codeemph{read()}.

Der schematische Aufbau solch einer Klasse:
%
\begin{lstlisting}
class limitedHerschelBulkley : public viscosityModel {
    tmp<volScalarField> nu() const {
        return nu_;
    }
    void correct() {
        nu_ = calcNu();
    }
    bool read(const dictionary& viscosityProperties) {
        ...
    }
};
\end{lstlisting}
%
In der Funktion \codeemph{correct()} wird hier eine private Funktion \codeemph{calcNu()} aufgerufen, die die Viskosität neu berechnet.
Das modifizierte Herschel-Bulkley Modell \todo{modHB referenzieren} kann dabei, dank der exzellenten Operatorüberladung von \openfoam{}, in einer sehr simplen Schreibweise implementiert werden:
%
\begin{lstlisting}
Foam::tmp<Foam::volScalarField>
calcNu() {    
    return
    (
     min(
         nu0_,
         ( tau0_ * (1.001-exp(-1*m_*sr()))  + k_ *rtone* pow( tone*sr(),n_ ) )
         /(max(sr(), dimensionedScalar ("VSMALL", dimless/dimTime, VSMALL)))
        )
    );
}
\end{lstlisting}
%
wobei \codeemph{sr()} die Scherrate ist und \codeemph{tau0\_}, \codeemph{k\_} und \codeemph{n\_} die Modellparameter sind.\\
Durch die Division durch das Maximum der Scherrate und \codeemph{VSMALL}, einer sehr kleinen Konstante, wird die Division durch Null vermieden.\\
Da das Herschel-Bulkley Modell eine unendliche Viskosität vorsieht wenn das Fluide nicht geschert wird, dies aber zu numerischen Problem führen kann, wird ausserdem die Viskositae;t nach oben hin durch die Nullviskosität \codeemph{nu0\_} beschränkt.
%
\subsubsection{Viskoelastische Modelle}
Der Löser \codeemph{viscoelasticFluidFoam} greift für die Berechnung des Schubspannungtensors auf eine separate Klasse \codeemph{viscoelasticModel} zurück, die ähnlich der \codeemph{transportModel} Klasse in \codeemph{simpleFoam} die Methoden \codeemph{divTau()} und \codeemph{correct()} zur Verfügung stellt.\\
Die Methode \codeemph{divTau()} berechnet dabei die Divergenz der Schubspannung $\T$ und implementiert gleichzeitig noch einen Teil der DEVSS Methodik indem sie die Split Stress Terme dazu addiert:
%
\begin{lstlisting}
Foam::tmp<Foam::fvVectorMatrix>
divTau(volVectorField& U) const {
    ...
    return
    (
        fvc::div(tau_/rho_, "div(tau)")
      - fvc::laplacian(etaPEff/rho_, U, "laplacian(etaPEff,U)")
      + fvm::laplacian( (etaPEff + etaS_)/rho_, U, "laplacian(etaPEff+etaS,U)")
    );
}
\end{lstlisting}

In der Methode \codeemph{correct()} wird bestimmt, wie sich der Schubspannungstensor von einem Zeitschritt zum nächsten ändert.\\
Diese Methode ist der Teil der Klasse, der an das gewünschte viskoelastische Modell angepasst werden muss. Das in dieser Arbeit verwendete White-Metzner Modell \eqref{eq:whiteMetznerModell} benötigt für das Update des Schubspannungstensors drei Schritte.
In einem ersten Schritt müssen die direkt von $\u$ abhängigen Variablen berechnet werden
\begin{lstlisting}
volTensorField L = fvc::grad(U());
volTensorField C = tau_ & L;
volSymmTensorField twoD = twoSymm(L);
volScalarField gamma = sqrt(2.0) * mag(symm(L));
\end{lstlisting}

In einem zweiten Schritt werden die scherratenabhängige Relaxationszeit $\lambda$ und Viskosität $\eta$ aktualisiert
\begin{lstlisting}
volScalarField etaPValue =
     max(
         etaMin_,
         min(
             etaMax_,
             ( tau0_ + K_ * rtone * Foam::pow( tone*gamma,n_ ) )
             /(max(gamma, dimensionedScalar ("VSMALL", dimless/dimTime, VSMALL)))
         )
     );

volScalarField lambdaValue = 
        lambdaInf_+(lambdaNull_-lambdaInf_) * 
                Foam::pow(
                1 + Foam::pow( L_* gamma,b_), 
                (nLambda_- 1)/b_
                );
\end{lstlisting}
%
Als letztes wird die Differentialgleichung mit dem Transport-Term implizit gelöst
%
\begin{lstlisting}
fvSymmTensorMatrix tauEqn
(
    fvm::ddt(tau_)
    + fvm::div(phi(), tau_)
    ==
    etaPValue/lambdaValue*twoD
    + twoSymm(C)
    - fvm::Sp(1/lambdaValue, tau_)
);
           
tauEqn.relax();
tauEqn.solve();
\end{lstlisting}
%
\subsection{Post processing}
Fue;r die Auswertung der Simulationen gibt es mit ParaView \todo{Quelle Paraview} und diversen \openfoam{} Utilities schon eine ganze Anzahl von Moe;glichkeiten.\\
In dieser Arbeit wurde zusae;tzlich noch die Moe;glichkeit genutzt eigene Runtime Functions zu schreiben, also Funktionen die schon wae;hrend der Simulation bestimme Werte berechnen und damit Einblick in die laufende Simulation ermoe;glichen.
%
\subsubsection{patchAverage}
Die Funktion \codeemph{patchAverage} ist eine vereinfachte Version der schon in \openfoam{} implementierten Runtime Function \codeemph{fieldValue}.\\
Sie dient dazu, in vom Nutzer bestimmten Zeitabstae;nden ein bestimmtes Feld ue;ber eine oder mehrere Randflae;chen der simulierten Geometrie zu mitteln und das Resultat in eine separate Datei zu speichern.\\
Ein Beispiel fue;r die Anwendung ist das Beobachten des mittleren Einlassdruckes bei festgelegter Einlassgeschwindigkeit in einer Stroe;mung.\\
Dazu mue;ssen im \fileemph{controlDict} folgende Zeilen hinzugefue;gt werden:
%
\begin{lstlisting}
functions
(
    DruckAmEinlass {
        type averageValue;
        functionObjectLibs (``libaverageValue.so'');
        patches (Einlass);
        outputControl timeStep;
        fields (p)
    }
);
\end{lstlisting}
%
Dadurch wird in jedem Zeitschritt die Funktion \codeemph{calcAverageValues} aufgerufen, die ue;ber alle Felder und Flae;chen iteriert und den Durchschnitt berechnet:
%
\begin{lstlisting}
forAll(fieldNames_, fieldi) {    
forAllConstIter(labelHashSet,patchSet_, iter) {
    label patchi = iter.key();
    // add this patch's area to the sum
    areas[fieldi] += sum(mesh.magSf().boundaryField()[patchi]);
    // multiply face area with face-value and sum up.
    avgVals[fieldi] +=
        sum(mesh.magSf().boundaryField()[patchi]*f.boundaryField()[patchi]);
    avgVals[fieldi] /= areas[fieldi];
}
return avgVals;
}
\end{lstlisting}
%
\subsubsection{torqueCalc}
Im Kapitel \ref{Kapitel:Korrektursimulation} wird beschrieben, wie die Viskositae;t des Moe;rtels anhand des Drehmomentes in einem Platte-Platte Rheometer bestimmt wird.\\
Die Berechnung dieses Drehmomentes kann zwar mit den schon vorhandenen Utilities in \openfoam{} durchgefue;hrt werden, allerdings ist das eher umstae;ndlich und es mue;ssen mehrere Funktionen aufgerufen werden.\\
Um das zu vermeiden, wurde die Funktion \codeemph{torqueCalc} geschrieben, die das Drehmoment in jedem Zeitschritt berechnet.
Dadurch ist nicht nur die Auswertung der Simulation einfacher, es kann durch eine Überprue;fung der Änderung des Drehmoments auch eine Konvergenzbedingung angegeben und die Simulation in jedem Zeitschritt beendet werden.

Die Implementierung wurde ebenfalls als runtime function umgesetzt. In der Funktion \codeemph{calcTorque()} wird das Drehmoment $M$ an den vom Nutzer bestimmten Randflae;chen $A_i$ berechnet:
\newnot{symbol:M}
\begin{equation}
    M = \sum_i \int_{A_i}r\tau\left( r \right)dA_i
\end{equation}
wobei 
\begin{equation}
    \tau\left( r \right) = \nu\left( r \right) \cdot \frac{\partial \u\left( r \right)}{\partial \n\left( r \right)}
\end{equation}
die Schubspannung an der Wand und $r$ der Abstand zum Rotationszentrum ist.

In \openfoam{} kann die Schubspannung mithilfe der schon implementierten Funktion \codeemph{snGrad()} sehr einfach berechnet werden, wobei die Viskositae;t erst spae;ter dazu multipliziert wird:
%
\begin{lstlisting}
wallShearStress = vel.boundaryField()[patchi].snGrad();
\end{lstlisting}

Der Abstand $r$ ist die X-Komponente eines lokalen Zylinder-Koordinatensystems
\begin{lstlisting}
cylindricalCS ccs
(
 "ccs",
 vector(0,0,0), // Origin
 vector(0,1,0), // Cylindrical axis
 vector(1,0,0)  // Vector with zero angle
);
\end{lstlisting}

Die Summe und das Integral ue;ber die vorgegebenen Flae;chen kann dann mittels einer for-Schleife implementiert werden:
\begin{lstlisting}
forAllConstIter(labelHashSet,patchSet_, iter)
{
 label patchi = iter.key();
 torque+= sum
 (
  ccs.localVector(wallShearStress)().
       component(vector::X) *
  transportModel.nu()().
       boundaryField()[patchi] * 
  ccs.localVector(mesh.C().boundaryField()[patchi])().
       component(vector::X) *
  mesh.magSf().boundaryField()[patchi]
 );
}
\end{lstlisting}

Die Konvergenzbedingung ist in der Funktion \codeemph{checkConvergence()} eingebaut.
Dabei wird fue;r das Berechnen der ersten und der zweiten Ableitung die finite Differenzen Methode benutzt:
%
\begin{eqnarray}
    \label{eq:torqueCalc:fd}
    \dot{m} & = & \frac{1}{2}\left(  m[t+1] - m[t-1] \right) \\
    \ddot{m} & = & \frac{1}{2}\left(  m[t+1] - 2 \cdot m[t] + m[t-1] \right) 
\end{eqnarray}
%
Die Werte $m[t]$ wurden dabei in einer \codeemph{fixedList} der Groe;sse drei gespeichert, in die jeweils periodisch die neuen Werte geschrieben werden
%
\begin{lstlisting}
label ip1 = convListIt_;
label i   = convList_.rcIndex(ip1);
label im1 = convList_.rcIndex(i);
scalar firstDeriv = (convList_[ip1]-convList_[im1])/2;
scalar secondDeriv = (convList_[ip1]-2*convList_[i]+convList_[im1]);

if (mag(firstDeriv) < convergence_ && mag(secondDeriv) < convergence_/100)
{
    Info << "Convergence reached, stopping simulation now" << nl;
    obr_.time().stopAt(Time::saWriteNow);
}
\end{lstlisting}
%
%
%\begin{lstlisting}
%scalar torque = calcTorque();

%convList_[convListIt_]=torque;
%// Get next index (periodic)
%convListIt_=convList_.fcIndex(convListIt_);
%\end{lstlisting}
%
\subsection{Betriebssystem und Hardware}
\label{Kapitel:Hardware}
\begin{todocontent}
    \1 Ubuntu
    \1 CSCS
    \1 Rothorn
    \1 Eiger
\end{todocontent}
