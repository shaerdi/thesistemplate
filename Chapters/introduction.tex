\section{Einleitung}
\label{Kapitel:Einleitung}
Die Firma Hilti AG stellt im Bereich der Dübel-Befestigungstechnik neben Kunststoff- und Metalldübeln auch chemische Dübel her, bei denen die Haltekraft durch das Aushärten eines Mörtels um eine Anker- oder Gewinde"-stange zustande kommt. \\
Dazu kommen unter anderem Zwei-Komponenten Mörtel auf Epoxy- und auf Harzbasis zum Einsatz. Die Aushärtung geschieht dabei aufgrund einer chemischen Reaktion, die startet sobald die beiden Komponenten in Kontakt kommen.\\
Die Mörtel sind rheologisch komplex und zeigen sowohl strukturviskoses als auch viskoelastisches Verhalten.% Diese Eigenschaften sind teilweise gewünscht, da sie ein Arbeiten auf der Baustelle erleichtern. Teilweise sind sie aber auch Nebeneffekte der Inhaltsstoffe, die für die chemische Reaktion des Mörtels gebraucht werden.

Die Hilti AG ist bekannt für zuverlässige und innovative Produkte, die jederzeit auf dem neusten Stand sind. Um diesen Standard auch weiterhin aufrecht zu erhalten, werden in der Forschungsabteilung auch numerische Simulationen verwendet.\\
Zur Simulation von Fluiden wurde bei Hilti AG bisher \cfx{} verwendet. Dieses Programm ist aber nicht in der Lage, viskoelastische Stoffe zu simulieren.% Deshalb wurde beschlossen, auf eine Alternative umzusteigen.

In dieser Arbeit soll als Alternative die freie Bibliothek \openfoam{} verwendet werden, um das Fliessverhalten der verwendeten Mörtel zu simulieren.

Dazu sollen passende konstitutive Gesetze ausgewählt werden, um die nicht-Newtonschen Eigenschaften der Mörtel abzubilden. Diese und die dazu notwendigen physikalischen Grundlagen sind in Kapitel \ref{Kapitel:Rheologie} beschrieben.\\
Die resultierenden Gleichungssysteme werden numerisch gelöst. Die dazu verwendeten iterativen Verfahren sind in Kapitel \ref{Kapitel:Numerik} aufgeführt.
Die notwendigen Tools und deren Implementation sollen dabei, sofern nicht schon darin enthalten, in \openfoam{} implementiert werden. In Kapitel \ref{Kapitel:Implementierung} wird die Vorgehensweise veranschaulicht.

In Kapitel \ref{Kapitel:Parameter} werden die variablen Parameter der konstitutiven Gesetze anhand von Messdaten eruiert und mit Simulationen verifiziert.\\

Zum Abschluss sollen in Kapitel \ref{Kapitel:Auspressgeraet} die Modelle durch die Simulation eines anwendungsnahen Versuchaufbaus validiert werden.
