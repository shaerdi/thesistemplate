\thispagestyle{empty}
\selectlanguage{ngerman}
\begin{abstr}
Einige von der Firma Hilti AG verwendete Mörtel zeigen ein rheologisch komplexes Fliessverhalten. Um ein besseres Verständnis der Mörtel zu ermöglichen, sollen diese mittels numerischen Simulationen untersucht werden.
Die Modellierung der rheologischen Eigenschaften erfordert die Wahl eines konstitutiven Gesetzes, mit dem ein Zusammenhang zwischen der am Fluid anliegenden Spannung und der resultierenden Verformung hergestellt werden kann.
Um ein solches Modell in einen Computercode einzubauen, wird ein flexibles Simulationsprogramm benötigt.
Aus diesem Grund wird in dieser Arbeit die Bibliothek \openfoam{} verwendet, die das Einbauen von eigenen numerischen Verfahren ermöglicht.


%In dieser Arbeit wird das rheologisch komplexe Fliessverhalten einiger von der Firma Hilti AG verwendeten Mörtel untersucht.\\
\begin{todocontent}
    \1 Konstitutive Modelle
    \1 Scherratenabhaengige Viskosität
    \1 Viskoelastizität
    \1 Modellparameter
    \1 Numerische Loesung
    \1 Verwendung Openfoam
    \1 Resultate
\end{todocontent}
\end{abstr}
%
\selectlanguage{english}
%
\begin{abstr}
    \begin{todocontent}
        \1 Übersetzung
    \end{todocontent}
\end{abstr}
%
\selectlanguage{ngerman}
